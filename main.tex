\documentclass[]{article} % Tipo de documento
\usepackage[utf8]{inputenc} % Uso de la codificacion UTF8
\usepackage{comment} % Incluir comentarios multilinea
\usepackage{amssymb} % Agregar simbolos matematicos
\usepackage{amsmath} % Agregar ecuaciones
\usepackage[spanish, mexico]{babel} % Idioma espanol, mexico para que diga tabla y no cuadro
\usepackage[colorlinks]{hyperref} % Hypervinculo
\usepackage{multirow} % Combinar filas en tablas
\usepackage{float} % Poder mover los flotantes
\usepackage{graphicx} % Agregar imagenes
\graphicspath{{img/}}

\begin{comment}
    comentario 
    comentario 
    aaaa
    parece que no se pueden agregar este tipo de comentarios en medio del documento
\end{comment}

\title{First document} % Titulo
\author{Exequiel Olivera \thanks{Tiranoexe}} % Autor
\date{\today} % Fecha

\begin{document}


    \begin{titlepage}
        \maketitle
    \end{titlepage}

    \begin{abstract}
        Este es un parrafo simple al comienzo el documento. Una breve introduccion al tema principal.
    \end{abstract}
    
        En este documento. algunos paquetes y parametros adicionales. fueron agregados. Hay un paquete de codificacion y parametros de tamano de fuente

        Esta linea comenzara un segundo parrafo. Y yo puedo romper \\ las lineas \\ y controlar en una nueva linea

        Algunos de los \textbf{grandes} descubrimientos en \underline{ciencia} fueron realizados por \textbf{\textit{accidente}}

        Algunos de los mejores descubrimientos en ciencia fueron realizados por \emph{accidente}\\
        %it y emph son para poner cursivas, pero emph lo hace a pesar de que se haya usado otro cambio en la fuente del texto
        

        \pagebreak
        \textbf{Listas}
        \begin{itemize}
            \item Las entradas individuales se indican con un punto negro, una llamada vineta.
            \item   El texto en las entradas puede ser de cualquier longitud.
        \end{itemize}

        \begin{enumerate}
            \item Las etiquetas consisten en numeros secuenciales.
            \item Los numeros comienzan en 1 con cada llamada al entorno de nenumeracion.
        \end{enumerate}

        \underline{Listas anidadas enumeradas}
        \begin{enumerate}
            \item Item de primer nivel
            \item Item de primer nivel
            \begin{enumerate}
                \item Item de segundo nivel
                \item Item de segundo nivel
                \begin{enumerate}
                    \item Item de tercer nivel
                    \item Item de tercer nivel
                    \begin{enumerate}
                        \item Item de cuarto nivel
                        \item Item de cuarto nivel 
                    \end{enumerate}
                \end{enumerate}
            \end{enumerate}
        \end{enumerate}

        \underline{Listas anidadas sin enumerar}
        \begin{itemize}
            \item Item de primer nivel
            \begin{itemize}
                \item Item de segundo nivel
                \begin{itemize}
                    \item Item de tercer nivel
                    \begin{itemize}
                        \item Item de cuarto nivel
                    \end{itemize}
                \end{itemize}
            \end{itemize}
        \end{itemize}

        \pagebreak
        \textbf{Expresiones matematicas}\\

        El conocido teorema de Pitagoras \(x^2 + y^2 = z^2\) demostro ser invalido para otros exponentes. Lo que significa que la siguiente ecuacion no tiene soluciones enteras: 

        \[ x^n + y^n = z^n\]

        % Se usa parentesis para la ecuacion en linea con el texto
        % Se usa corchete para la ecuacion en una nueva linea

        \underline{En linea con el texto}
        \begin{itemize}
            \item En fisica, se establece la equivalencia masa-energia por la ecuacion $E=mc^2$, descubierta en 1905 por Albert Einstein.

            \item En fisica, se establece la equivalencia masa-energia por la ecuacion \(E=mc^2\), descubierta en 1905 por Albert Einstein.

            \item En fisica, se establece la equivalencia masa-energia por la ecuacion \begin{math}E=mc^2\end{math}, descubierta en 1905 por Albert Einstein.
        \end{itemize}

       \underline{Simbolos matematicos}
       \begin{itemize}
        \item Letras griegas: $\alpha, \beta, \gamma, \rho, \sigma, \delta, \epsilon, etc.$

        \item Operadores binarios: $\times, \otimes, \oplus, \cup, \cap, etc.$ 

        \item Operadores de relaciones: <, >, $\subset, \supset, \subseteq, \supseteq, etc.$
        
        \item Otros: $\int, \oint, \sum, \prod, etc.$\\
        \end{itemize}

        \pagebreak
        \textbf{Ecuaciones}

        \begin{equation} \label{eq1}
            \begin{split}
            A & = \frac{\pi r^2}{2} \\
             & = \frac{1}{2} \pi r^2
            \end{split}
        \end{equation}

        % Debe ser incluida dentro del ambiente equation
        % Se puede usar equation* si no se quiere enumerar
        % Split para dividir la ecuacion en partes, que van a ser alineadas automaticamente
        % El ampersand (&) sirve para establecer los puntos de respecto a los cuales se alinearan verticalmente las ecuaciones

        \underline{Ecuaciones de una sola linea}
        \begin{equation} \label{eu_eqn}
            e^{\pi i} - 1 = 0
        \end{equation}

        La hermosa ecuacion \ref{eu_eqn} is known as the Euler equation.\\

        % \label es para etiquetar y luego poder referirse a ella o su numeracion con \ref
        
        \underbar{Ecuaciones de varias lineas}
        \begin{multline*}
            p(x) = 3x^6 + 14x^5y + 590x^4y^2 + 19x^3y^3\\
            -12x^2y^4 - 12xy^5 + 2y^6 - a^3b^3 - 48x4^3y
        \end{multline*}

        % Se usa cuando la ecuacion no entra en una sola linea

        \underbar{Division y alineacion de ecuaciones}
        \begin{align*} 
            2x - 5y &=  8 \\ 
            3x + 9y &=  -12
        \end{align*}

        % Se usa \align si hay varias ecuaciones que deben alinearse verticalmente
        % Se toma a los operadores <, > y = como puntos de alineacion vertical

        \begin{align*}
            x&=y           &  w &=z              &  a&=b+c\\
            2x&=-y         &  3w&=\frac{1}{2}z   &  a&=b\\
            -4 + 5x&=2+y   &  w+2&=-1+w          &  ab&=cb
            \end{align*}

        % Se vuelve a usar el & para determinar los puntos de alineacion de las ecuaciones, en este caso, formando 3 columnas 

        \underbar{Agrupar y centrar ecuaciones}
        \begin{gather*} 
            2x - 5y =  8 \\ 
            3x^2 + 9y =  3a + c
        \end{gather*}
        
        % Se usa para mostrar una serie de ecuaciones consecutivas, centradas y sin alinear.

        \underbar{Cambiar el estilo y alineacion default}

        Pasar de
        \[
             a_0+{1\over a_1+
                  {1\over a_2+
                    {1 \over a_3 + 
                       {1 \over a_4}}}}
        \]

        a

        \[
            a_0+{1\over\displaystyle a_1+
                {1\over\displaystyle a_2+
                    {1 \over\displaystyle a_3 + 
                    {1 \over\displaystyle a_4}}}}
            \]
        
        % \textstyle aplica el estilo usado para la tipografia matematica en los parrafos
        % \displaystyle: aplica el estilo utilizado para la tipografía matemática en líneas por sí mismas
        % \scriptstyle: aplica el estilo usado para subíndices o superíndices
        % \scriptscriptstyle: aplica el estilo usado para subíndices o superíndices de segundo orden

        \underbar{Coeficiente binomial o numeros combinatorios}

        El coeficiente binomial, \(\binom{n}{k}\), se define por la expresion:
        \[
            \binom{n}{k} = \frac{n!}{k!(n-k)!}
        \]

        % Como el \frac, el \binom divide la expresion y las acomoda verticalmente

        \underline{Estilo de fracciones con texto}

        \[\frac{\text{numerador}}{\text{denominador}}\]

        % Usando \verb o \text{...}
        % Sin usar \verb o \text el resultado es este: 

        \[\frac{numerator}{denominator}\]

        \pagebreak
        \textbf{Tablas}\\

        \begin{tabular}{|c|c|c|c|}
            \hline
            Super & Transporte & Facultad & Extra \\
            \hline
            5400 & 800 & 1400 & 4000 \\
            \hline
            TOTAL & 11600 \\
            
        \end{tabular}
  
        % Despues del tabular se pone c(center), l(left), r(right) o p(parragraph){con una distancia en cm}

        \begin{table}[h] % La h para que la tabla aparezca justo aqui, y no en top o bottom
            \centering
            \caption{Leyenda de la tabla.}
            \begin{tabular}{|c|c|c|c|}
                \hline
                Super & Transporte & Facultad & Extra \\
                \hline
                5400 & 800 & 1400 & 4000 \\
                \hline
                TOTAL & 11600 \\
            \end{tabular}
            \label{tab:datos}
        \end{table}

        \begin{table}[h] % La h para que la tabla aparezca justo aqui, y no en top o bottom
            \centering
            \caption{Horario de cursado.}
            \begin{tabular}{|c|c|c|c|c|}
                \hline
                Lunes & Martes & Miercoles & Jueves & Viernes \\
                \hline
                - & Discreta & - & Discreta & - \\
                \hline
                Laboratorio & Algoritmos & Analisis & Algoritmos & Analisis \\
            \end{tabular}
            \label{tab:datos2}
        \end{table}
        
        % Para "arreglar" que en la tabla 1 hay columnas faltantes, vamos a combinar columnas

        \begin{table}[h]
            \centering
            \caption{Tabla con columnas combinadas.}
            \begin{tabular}{|c|c|c|c|}
                \hline
                Super & Transporte & Facultad & Extra \\
                \hline
                5400 & 800 & 1400 & 4000 \\
                \hline
                TOTAL & \multicolumn{3}{|c|}{11600} \\ 

                % El 3 indica la cantidad de columnas agrupadas en la celda
                % El |c| indica que el texto debe estar centrado y con lineas verticales
                % La ultima llave indica el valor de la columna
                
            \end{tabular}
            \label{tab:datos3}
        \end{table}

        \begin{table}[H] %  'H' y no 'h' ya que sin el paquete float la tabla se iba a una nueva hoja
            \centering
            \caption{Tabla con filas combinadas.}
            \begin{tabular}{|c|c|c|c|}
                \hline
                Super & Transporte & Facultad & Extra \\
                \hline
                5400 & 800 & \multirow[c]{2}{*}{1400} & 4000 \\
                \cline{1-2}
                \cline{4-4}
                3400 & 600 &  & 2000 \\
                \hline
                TOTAL & \multicolumn{3}{|c|}{11600} \\

                % El c indica que el texto debe estar centrado
                % El 2 indica la cantidad de filas agrupadas en la celda
                % El * indica que tenga el ancho de toda la columna
                % La ultima llave indica el valor de la columna, se puede poner un ancho especifico con width
                % Para sacar la linea que queda en medio de las filas, al \hline lo reemplazo por \cline{1-2} y \cline{4-4}
                % Indicando que la linea debe ir desde la columna 1 a la 2 y tambien en la 4, siendo la 3 en la que las filas se combinan
                
            \end{tabular}
            \label{tab:datos4}
        \end{table}

        Ver tablas: 
            Tabla \Ref{tab:datos}, tabla \Ref{tab:datos2}, tabla \Ref{tab:datos3}, tabla \Ref{tab:datos4} 
            % Referencia a la tabla con \label e hypervinculo
         
        \pagebreak
        \textbf{Imagenes}

        % Primero agregar el paquete graphicx
        % Crear una carpeta nueva para guardar las imagenes (img)
        % Incluir una linea de codigo que indique al compilador la ruta de acceso a la carpeta de las imagenes ( \graphicspath{{img/}} )
        % Agregar el comando \includegraphics{NOMBREDEIMAGEN}

        \includegraphics{dd70b1b87d300d20fcfaf566f13de977.jpg}

        \underline{Ajustar el tamano de la imagen}

        % Usando [scale]
        % Con [scale=0.5] se reduce el tamano de la imagen a la mitad
        \includegraphics[scale=0.5]{dd70b1b87d300d20fcfaf566f13de977.jpg}

        % Tambien se puede indicar el ancho y alto especificos usando [width, height]
        % Se pueden poner los 2 valores, o solo 1 de los 2, para mantener la proporcion de la imagen
        \includegraphics[width=5cm, height=7cm]{dd70b1b87d300d20fcfaf566f13de977.jpg}

        % Para ajustar la imagen al ancho del texto se usa [width=\textwidth]
        \includegraphics[width=\textwidth]{dd70b1b87d300d20fcfaf566f13de977.jpg}\\

        \underline{Posicionamiento de la imagen}

        % Incluir entorno figure 
        \begin{figure} 
            \includegraphics[width=\textwidth]{dd70b1b87d300d20fcfaf566f13de977.jpg}
        \end{figure}

        % Usando [t] (top)e n el begin, la foto se pone al comienzo de la pagina
        % Usando [b] (bottom) en el begin, la foto se pone al final de la pagina, o a la siguiente pagina
        % Usando [h] (here) en el begin, la foto se pone justo donde aparece el entorno de la pagina
\end{document}


\begin{comment}
    Cambiar la numeracion: \renewcommand{\labelenumi}{\(TIPODENUMERACION){enumi}}
    Despues del labelnum y el enum va en romano el nivel, ej enumii (2do nivel)
    
    Tipos de numeracion:
    \alph: Letra minuscula (a, b, c...)
    \ALPH: Letra mayuscula (A, B, C...)
    \arabic: Numero arabigo (1, 2 ,3...)
    \roman: Numero romano en minuscula (i, ii, iii...)
    \Roman: Numero romano en mayuscula (I, II, III...)

    Cambiar el simbolo en itemize: \renewcommand*{\laberlitemi}{$\TIPODESIMBOLO$}
    Despues del laberlitem tambien va el nivel en romano
    
    Tipos de simbolos:
    $\blacksquare$: cuadrado negro
    $\square$: cuadrado vacio

\end{comment}
